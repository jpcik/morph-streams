\section{Introduction}
\label{intro}

Recent advances in wireless communications and sensor technologies have opened the way for deploying networks of
interconnected sensing devices capable of ubiquitous data capture, processing and delivery. Sensor network deployments
are expected to increase significantly in the upcoming years because of their advantages and unique features. Tiny
sensors can be installed virtually anywhere and still be reachable thanks to wireless communications. Moreover, these
devices are inexpensive and can be used for a wide range of applications such as security surveillance, traffic
control, environmental monitoring, healthcare provision, industrial monitoring, etc.\

One of the means to access streaming data sources coming from sensor networks is through query
processors~\cite{Madden_05,Arasu_06a,Galpin_09} that handle streaming data (which differs significantly from classical
stored data, as it is potentially infinite and transient, with tuples being constantly added) and support declarative
continuous query languages (for which query results are updated regularly as time passes~\cite{Terry_92}).

%A key requirement for sensor network client applications is to be able to access heterogeneous published data sources.
%Consumers such as mashups need to perform queries over streaming sources published using web services and query
%endpoints.

%Sensor network characteristics also raise several challenges for the research community, such as the integration of the data collected by these sensors. Client applications such as mashups require a suitable platform to access and combine heterogeneous data coming from different sensor networks and other sources such as databases. Furthermore, these applications require accessing this data in terms of a uniform schema that hides the diverse and internal data representations of each source.\
In the context of the Semantic Web vision, several initiatives that aim at providing semantic access to traditional
(stored) data sources have been launched in the past years. Most of the existing approaches attempt to provide mappings
between the elements in the relational and ontological models \cite{Sahoo_09}, as we will describe in the Background section.%\ref{previousworks}. 
However, to the best of our knowledge, similar solutions for streaming data mapping and querying
using ontology-based approaches have not been explored yet in depth.
%Semantic technologies and ontologies have been successfully used in the latest years for data integration solutions, particularly in the area of databases. The problem of integrating these sources involves not only granting access to information but also providing mechanisms for interpreting and processing the data consistently, overcoming syntactical and semantic heterogeneity.\\
%Although proposals for streaming data access, ontology-based integration and query languages for RDF streams have been proposed and implemented, to the best of our knowledge there is no framework yet that integrates these different pieces into a coherent solution for integration of distributed heterogeneous streaming and static data sources through ontological models.
%In this paper we focus on the first step to achieve this level of integration, that is providing ontology-based access to continuous data sources, including sensor networks.
%Semantic technologies and ontologies have been successfully used in the latest years for data integration solutions, particularly in the area of databases [Wache, halevy]. The problem of integrating these sources involves not only granting access to information but also providing mechanisms for interpreting and processing the data consistently, overcoming syntactical and semantic heterogeneity.\\

In this paper we focus on providing ontology-based access to streaming data sources, including sensor networks, through
declarative continuous queries. This constitutes a first step towards a framework for the integration of distributed
heterogeneous streaming and stored data sources through ontological models and to the provision of Linked Data for
streams~\cite{LePhuoc_09,Page_09,Sequeda_09}. The paper is organised as follows: in the Background section %\ref{previousworks} 
we introduce previous work. The foundations of our approach are explained in the Ontology-based Streaming Data Access section. %\ref{approach}. 
In the Extensions Syntax section
%\ref{syntax} 
we present the syntactic extensions for RDF stream SPARQL operators, and \rtwoo\ stream-to-ontology
mappings. The semantics of these extensions are detailed in the Streaming Extensions Semantics section %\ref{semanticsstreaming} 
and a first implementation of the execution of the streaming data access approach is explained in the Implementation and Walkthroug section. % \ref{execution}. 
Finally we present the conclusions and future work.
%- introduce streaming query languages%
%- reformulate RDF based access to sources
%- integrate SNs, event streams, DBs, -> need
%-first step-> expose through RDF
%test cite\cite{Gangemi_2006a}
