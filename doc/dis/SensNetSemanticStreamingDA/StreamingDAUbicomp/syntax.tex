\section{Extensions Syntax}
\label{syntax}

In this section we introduce the language extensions to \linebreak SPARQL, for RDF stream management, and to \rtwoo\, for the
definition of stream-to-ontology mappings.

\subsection{Streaming Extensions to SPARQL}
\label{streamingsparqlsyntax}

As shown previously %in Section \ref{rdfstreams}
, C-SPARQL introduces extensions for the support of RDF streams.\ %, mainly time and tuple based windows and registration of queries and streams \cite{Barbieri_2010}.
The language is expressive enough \linebreak to support most of the constructs we require, including time and tuple windows,
aggregates, query and stream registration, and joins between streaming and stored data. Moreover, as in C-SPARQL, we
follow the approach of applying windows to streaming data and afterwards applying standard operators over the resulting
non-streaming output \cite{DellaValle_09}. This slightly extended C-SPARQL variation is named \sparqlstr\ in the rest
of the paper.

Just as in \cite{Barbieri_2010} we define an RDF Stream as a sequence of pairs $(T_i,\tau_i)$ where $T_i$ is an RDF
triple $\langle s_i,p_i,o_i \rangle$ and $\tau_i$ is a timestamp.
%A Stream is specified using the \textsf{STREAM} clause in the \textsf{FROM} part of the query. The stream itself is identified with an \textit{IRI} (Internationalized Resource Identifier) that uniquely identifies it. Windows are specified using square brackets \textsf{[ ]} and the \textsf{RANGE} keyword for specifying the time based window width.For instance the expression \textsf{[RANGE 10 h]} applied to a stream will take the triples registered in the last 10 hours.
In addition to the time width specified with the \textsf{RANGE} keyword, we introduce the possibility of specifying initial and final time boundaries for windows, using the \textsf{TO} keyword. We also add the \textsf{NOW} keyword, used to denote the current timestamp. Using these additions it is possible to specify more complex time ranges such as intervals in the past. The general form of the window range is \textsf{[RANGE $t_i$ TO $t_f$]}, where $t_i$ and $t_f$ are the lower and higher time boundaries respectively. Both boundaries are of the form \textsf{NOW-}$t$, where $t$ is a number of time units. For example the window \textsf{[RANGE NOW-2 d TO NOW-1 d]} will take all triples registered between one and two days ago. %The higher boundary can be ignored if we want it to be the current timestamp (i.e. TO NOW-0).
A slide parameter can be specified using the \textsf{STEP} keyword and the time interval for the sliding window creation.\ %, e.g. [RANGE 30 m STEP 5m] will take triples registered 30 minutes ago, every 5 minutes.
Triple based windows are of the form \textsf{[ROWS N]} where \textsf{N} is the number of triples to be taken.

%add query registration
%add stream registration
%add agregates


%First we introduce the extensions to SPARQL for RDF streams. Although we already introduced

%Here i explain first the extensions to SPARQL. Based on C-SPARQL. mention the additions necessary, any time window definition, rstream operator. Aggregates for the future. here i put some syntax samples and explicative description of what it means

\subsection{Streaming Extensions to \rtwoo}
\label{streamingr2osyntax}

The mapping document that describes how to transform the data source elements to ontology elements is written in the
\stwoo\ mapping language, an extended version of \rtwoo. As it is explained in \cite{Barrasa_04}, \rtwoo\ includes a
section in the mapping document that describes the database tables and columns, \textsf{dbscehma-desc}. In order to
support streams, \rtwoo\ has been extended to also describe the data stream schema. A new component called
\textsf{streamschema-desc} has been created, as in the following example:

\lstdefinelanguage{R2O}{
morekeywords={streamschema,desc,name,has,stream,streamType,documentation,timestamp, keycol,nonkeycol,columnType,conceptmap,def,uri,as,virtualStream,described,by,
attributemap,has,column,applies,if,operation,dbrelationmap,toConcept,joins,via,condition},
sensitive=true,%
morecomment=[l]\#,%
morestring=[b]',%
}
\lstdefinestyle{R2OStyle}{basicstyle=\sffamily\scriptsize,
                        %keywordstyle=\lstuppercase,
                        emphstyle=\itshape,
                        showstringspaces=false,
                        tabsize=2,
                        }
\begin{lstlisting}[style=R2OStyle,language=R2O,frame=none]
streamschema-desc
    name CoastalSensors
    has-stream SensorWaves
        streamType pushed
        documentation "Wave measurements"
        keycol-desc measurementid
            columnType integer
        timestamp-desc  measuretime
            columnType datetime
        nonkeycol-desc  measureheight
            columnType float
        nonkeycol-desc measuretemperature
            columnType float
\end{lstlisting}
The description of the stream is similar to a table. An additional attribute \textsf{streamType} has been added, it denotes the kind of stream in terms of data acquisition. It can be a sensed stream, i.e. pull based arriving at some acquisition rate. Or it can be pushed, arriving at some potentially variable and/or unknown rate. Relations can also be specified just like tables in \stwoo. In the same way as key and non key attributes are defined, a new \textsf{timestamp-desc} element has been added to provide support for declaring the stream timestamp attribute.
For the class and property mappings, the \rtwoo\ existent definitions can be used for stream schemas just as it was for relational schemas. This is specified in the \textsf{conceptmap-def} element:
\begin{lstlisting}[style=R2OStyle,language=R2O,frame=none]
conceptmap-def Wave
    virtualStream <http://virtualStreamIRI>
    uri-as
        concat(SensorWaves.measurementID)
    applies-if
        <cond-expr>
    described-by
        attributemap-def hasHeight
            virtualStream <http://virtualStreamIRI>
            operation constant
                has-column SensorWaves.measureheight
\end{lstlisting}
In addition, although they are not explicitly mapped, the timestamp attribute of stream tuples could be used in some of the mapping definitions, for instance in the URI construction (\textsf{uri-as} element).
Finally, it has been seen that at the moment of generating a SPARQL streaming query, an RDF Stream IRI is expected along with the window parameters. In this case the RDF Stream is virtual and its IRI can be specified in the \stwoo\ mapping using the \textsf{virtualStream} element. It can be specified at the \textsf{conceptmap-def} level or at the \textsf{attributemap-def} level.

%As it has been seen, R2O requires some changes to support creating mapping documents for stream schemas. It is expected that common and simple mappings in the R2O language can be defined using the ODEMapster tool, enhanced with streaming data support.



%Then I describe the extensions to R2O, how i specify a stream, with keys, i explain that no time windows needed in the mapping definitions. but taken into account during execution and transformation. Some example of mapping. Base somehow on the odba italians stuff
